\documentclass[10pt]{article}


\title{Foundations \\ \large Independent Society of Knowledge White Paper}
\author{Amir H. Ebrahimnezhad}
\date{last edited: \today}

\begin{document}
\maketitle
\section{Definition}\label{sec:definition}
Independent Society of Knowledge is a community searching for an alternative academia that would accompany the current science and research principles, in a more dynamic, modern, and collaborative way.
The core principles of ISK are Decentralization, Collaboration, and Open-Access.

\textbf{Defining Decentralization:} Unlike most academics, ISK is not bound to a physical place, nor it is authorized by an authority.
Be it the government of a country, or funders of the organization.
It is decentralized in the sense that projects can evolve freely, where the core members, and contributors decide it to go.
This aspect of ISK in my opinion helps the organization finds its path through science and research, without many restrictions that current acedemia faces.

\textbf{Defining Collaboration:} In academia, since it is mostly localized in countries, people contribute more to the projects they find in their local academic.
ISK on the other hands visions the idea of collaborative knowledge growth, where people from different parts of the world can be organized under same values and interests to help with projects.
This approach to science is unique, but we have seen it in places like software engineering, where people collaborate on open-source projects.
ISK is similarly to open-source movement in this aspect.
It introduces the world of science to collaborative works in an open-source fashion.

\textbf{Defining Open-Access:} Knowledge should be accessible for all, and any should be able to contribute to it.
Academics define the essence of certification, and degrees, but lack good solutions to make these certifications valuable.
Grading, Assignments and many other features of academics are redundant for scientific work (research to be specific).
At ISK we value people by their contribution to knowledge, not by numbers they gained on the old-fashioned academia.
Open-Access for ISK means accessibility of knowledge to be gained, and to be contributed to, by anyone that values the same goals.
This means that although we are interested in a free knowledge, we do not attempt to provide it to people who don't value the freedom we advocate.
We also aim to provide knowledge for all without the loss of the right of people who contributed to it.

These are the cornerstones of The Independent Society of Knowledge.
It is in our opinion the true "Decentralized Academia".

\section{Executives, Members, and Collaborators}
In spite of the collaborative manner of ISK, we should stay organized and manageable from time to time.
Since the execution of projects that are large scale need an organized manner of collaboration, rather than random aims of each contributor.
Therefore, ISK is charted with different roles for projects, to help the organization grow in the direction the members decide to.
There are three main contributors for projects at ISK, Executives, Members, and Collaborators.

\textbf{Defining Executives:} The core members of ISK are executives, These people are selected from the pool of members decided by the current executives (which at the beginning is just me).
The role of the executives is to ensure that the values of ISK and the path it is taking is the one that the members want it to.
They are similar to team-managers, CEOs, and Chief Architects of the projects.

\textbf{Defining Members:} Although contributing to ISK's projects is open for all, we would invite some people that are deemed valuable to this society, to be our members.
Members are selected people that are invited by the society (Older members and Executives).
These people have a weighted vote on ISK's polls, where the society decides its next actions.

\textbf{Defining Collaborators:} Other than the two types above, anyone can enjoy contributing to open knowledge that ISK visions.
Any person that contributes to ISK on any level, is a collaborator.
He/She is welcomed to enjoy the free society we are aiming to build.

\section{Licenses}
For our open-source products, we adopt the GNU General Public License version 2.0 (GPL-2.0). This ensures that our work remains open-source and freely available for anyone to use, modify, and distribute. The GPL-2.0 license guarantees that any derivative works also remain open and contribute back to the community, aligning with our core principles of decentralization, collaboration, and open-access.

For our source available products we adopt a license that supports our vision of open knowledge while restricting commercial use. These products will be licensed under the Creative Commons Attribution-NonCommercial 4.0 International (CC BY-NC 4.0) license. This license allows others to remix, tweak, and build upon our work non-commercially, as long as they credit us and license their new creations under the identical terms.



\end{document}
